%%%%%%%%%%%%%%%%%%%%%%%%%%%%%%%%%%%%%%%%%%%%%%%%%%%%%%%%%%%% 
%% This Beamer template was created by Cameron Bracken.
%% Anyone can freely use or modify it for any purpose
%% without attribution.
%% 
%% Last Modified: January 9, 2009
%% 

\documentclass[xcolor=x11names,compress]{beamer}

%% General document %%%%%%%%%%%%%%%%%%%%%%%%%%%%%%%%%%
\usepackage{graphicx}
\usepackage{tikz, amsmath, amsfonts, amsthm, amssymb, algorithm2e}
\usepackage{hyperref}
\hypersetup{urlcolor=cyan}
\usetikzlibrary{decorations.fractals}
\beamertemplatenavigationsymbolsempty
%%%%%%%%%%%%%%%%%%%%%%%%%%%%%%%%%%%%%%%%%%%%%%%%%%%%%% 


%% Beamer Layout %%%%%%%%%%%%%%%%%%%%%%%%%%%%%%%%%%
\useoutertheme[subsection=false,shadow]{miniframes}
\useinnertheme{default}
\usefonttheme{serif}
\usepackage{palatino}

\setbeamerfont{title like}{shape=\scshape}
\setbeamerfont{frametitle}{shape=\scshape}

\setbeamercolor*{lower separation line head}{bg=DeepSkyBlue4} 
\setbeamercolor*{normal text}{fg=black,bg=white} 
\setbeamercolor*{alerted text}{fg=red} 
\setbeamercolor*{example text}{fg=black} 
\setbeamercolor*{structure}{fg=black} 

\setbeamercolor*{palette tertiary}{fg=black,bg=black!10} 
\setbeamercolor*{palette quaternary}{fg=black,bg=black!10} 

\renewcommand{\(}{\begin{columns}}
  \renewcommand{\)}{\end{columns}}
\newcommand{\<}[1]{\begin{column}{#1}}
  \renewcommand{\>}{\end{column}}
\setbeamercovered{transparent}
\SetKwFor{Class}{Class}{}{end}
\SetKwFor{Method}{Method}{}{end}
\SetKw{Emit}{Emit}
\SetKw{Size}{Size}
\SetKw{Clear}{Clear}
\SetKw{Near}{Neighbors}
\SetKw{Flush}{Flush}
\SetKw{Sum}{ElementSum}
%%%%%%%%%%%%%%%%%%%%%%%%%%%%%%%%%%%%%%%%%%%%%%%%%% 




\begin{document}


%%%%%%%%%%%%%%%%%%%%%%%%%%%%%%%%%%%%%%%%%%%%%%%%%%%%%% 
%%%%%%%%%%%%%%%%%%%%%%%%%%%%%%%%%%%%%%%%%%%%%%%%%%%%%% 
\section{\scshape Introduction}
\begin{frame}
  \title{Using Local Aggregation in MapReduce Jobs}
  \author{
    Andrew Johnson
  }
  \date{
    \begin{tikzpicture}[decoration=Koch curve type 2] 
      \draw[DeepSkyBlue4] decorate{ decorate{ decorate{ (0,0) -- (3,0) }}}; 
    \end{tikzpicture}  
    \\
    \vspace{1cm}
    March 25, 2013
  }
  \titlepage
\end{frame}

\begin{frame}{Who am I?}
  \begin{itemize}
  \item Software engineer at Explorys
  \item CWRU grad in Computer Science, May 2012
  \end{itemize}
\end{frame}

\subsection{Overview}

\begin{frame}{Overview}
  \begin{itemize}
  \item<1,2,3> You've got your cluster set up and configured well.
  \item<2,3> You've written your MapReduce jobs.
  \item<3> But they are just not as fast as you'd like
  \item<4> How do you take your MapReduce jobs to the next level?
  \end{itemize}
\end{frame}

\section{\scshape Know Thyself}

\subsection{Know Thyself}
\begin{frame}{Know Thyself}
  \begin{itemize}
  \item<1> Understand your data!
  \item<2,3> Understand the current performance characteristics of your jobs!
  \item<3> Understand how much harder you can push your cluster!
  \end{itemize}
\end{frame}

\begin{frame}{Understanding Your MapReduce Jobs}
  \begin{itemize}
  \item<1,2> You need to have some kind of monitoring in place (e.g. Ganglia)
  \item<2,3> Your first task is to find out what is limiting the
    performance of your job.
  \item<3> Watch resource utilization during the time the job is
    running; is it CPU, IO, or memory bound?
  \end{itemize}
\end{frame}

\begin{frame}{Disclaimer}
  \begin{itemize}
  \item<1,2> The algorithmic techniques in this presentation reduce network
    utilization at the cost of increased CPU and memory utilization in
    the map tasks.
  \item<2> While trying them out, continue to monitor resource
    utilization. \bf{Bad things can happen if you don't!}
  \item<3,4> I cannot guarantee that what I'm about to show you will
    actually improve the performance of your jobs.
  \item<4> Treat implementing these techniques as a science experiment.
  \item<5> That said, I've had good results with these techniques.
  \end{itemize}
\end{frame}

\section{\scshape In-Mapper Combining}

\subsection{Overview}
\begin{frame}{Word Count}
  \begin{algorithm}[H]
    \Class{Mapper}{
      \Method{Map(docid \emph{a}, doc \emph{d})}{
        \ForEach{term t $\in$ doc d} {
          \Emit{(term t, count 1)}
        }
      }
    }
    \Class{Reducer}{
      \Method{Reduce(term \emph{t}, counts $[c_1,c_2,\ldots]$)}{
        sum$\ \leftarrow 0$\\
        \ForEach{count c $\in$ counts $[c_1,c_2,\ldots]$} {
          sum$\ \leftarrow\ $ sum$\ +\ $c
        }
        \Emit{(term t, count sum)}
      }
    }
  \end{algorithm}
\end{frame}

\subsection{Combiners}

\begin{frame}
  \frametitle{Combiners}
  \begin{itemize}
  \item<1> After monitoring this job, we want to improve its performance
  \item<2> Why not try a combiner?
  \item<3,4> No guarantee how often a combiner is called, if at all
  \item<4> Using combiner still requires materializing all key-value pairs
  \end{itemize}
\end{frame}

\subsection{Local Aggregation}

\begin{frame}
  \frametitle{In-Mapper Combining}
  \begin{itemize}
  \item<1> Move the combiner functionality into the mapper
  \item<2,3> Provides total control over how and when combining takes place
  \item<3> Materialize the minimum number of key-value pairs
  \end{itemize}
\end{frame}
\begin{frame}
  \frametitle{Word Count}
  The reducer is the same as before
  \begin{algorithm}[H]
    \Class{Mapper}{
      \Method{Setup}{
        H$\ \leftarrow\ $ new Map$<$term$\ \rightarrow\ $ count$>$
      }
      \Method{Map(docid \emph{a}, doc \emph{d})}{
        \ForEach{term t $\in$ doc d} {
          H\{t\}$\ \leftarrow\ $ H\{t\}$\ +\ $1
        }
      }
      \Method{Cleanup}{
        \ForEach{term t $\in$ H} {
          \Emit{(term t, count H\{t\})}
        }
      }
    }
  \end{algorithm}
\end{frame}

\subsection{Potential Issues}
\begin{frame}
  \frametitle{Potential Issues}
  \begin{itemize}
  \item<1,2> Preserving state across calls to \emph{Map} may cause bugs
  \item<2> Order-dependent bugs are especially pernicious
  \item<3,4> Memory usage is fundamental scalability issue
  \item<4> Don't forget your monitoring!
  \end{itemize}
\end{frame}

\begin{frame}
  \frametitle{Dealing with Memory Usage}
  \begin{itemize}
  \item<1,2> Know your data and your grid!
  \item<2> Memory usage may not be a problem
  \item<3,4> Periodically flush intermediate key-value pairs
  \item<4> Empirically discover balance between memory usage and
    network utilization
  \end{itemize}
\end{frame}

\begin{frame}
  \frametitle{Word Count}
  \scriptsize
  \begin{algorithm}[H]
    \Class{Mapper}{
      \Method{Setup}{
        H$\ \leftarrow\ $ new Map$<$term$\ \rightarrow\ $ count$>$\\
        BufferSize$\ \leftarrow n$ 
      }
      \Method{Flush}{
        \ForEach{term t $\in$ H} {
          \Emit{(term t, count H\{t\})}
        }
        \Clear{(H})
      }
      \Method{Map(docid \emph{a}, doc \emph{d})}{
        \ForEach{term t $\in$ doc d} {
          H\{t\}$\ \leftarrow\ $ H\{t\}$\ +\ $1
        }
        \If{\Size{(H)} $\ \geq\ $ BufferSize} {
          \Flush
        }
      }
      \Method{Cleanup}{
        \Flush
      }
    }
  \end{algorithm}
\end{frame}

\section{\scshape Pairs and Stripes}

\subsection{Overview}

\begin{frame}
  \frametitle{Overview}
  \begin{itemize}
  \item<1> In-mapper combining is not the only local aggregation technique
  \item<2> For more complex problems, consider changing the
    intermediate key-value types
  \end{itemize}
\end{frame}

\subsection{Pairs}
\begin{frame}
  \frametitle{Pairs Approach to Word Co-Occurrence}
  \small
  \begin{algorithm}[H]
    \Class{Mapper}{
      \Method{Map(docid \emph{a}, doc \emph{d})}{
        \ForEach{term w $\in$ doc d} {
          \ForEach{term u $\in$ \Near{(w)}}{
            \Emit{(pair (w,u), count 1)}
          }
        }
      }
    }
    \Class{Reducer}{
      \Method{Reduce(pair p, counts $[c_1,c_2,\ldots]$)}{
        sum$\ \leftarrow 0$\\
        \ForEach{count c $\in$ counts $[c_1,c_2,\ldots]$} {
          sum$\ \leftarrow\ $ sum$\ +\ $c
        }
        \Emit{(pair p, count sum)}
      }
    }
  \end{algorithm}
\end{frame}

\begin{frame}
  \frametitle{Improvements}
  \begin{itemize}
  \item<1,2> Can apply in-mapper combining
  \item<2> There will still be a very large number of intermediate
    key-value pairs
  \item<3,4> What do you do if the shuffle phase is still the bottleneck?
  \item<4> Change the structure of your key-value pairs!
  \end{itemize}
\end{frame}

\subsection{Stripes}

\begin{frame}
  \frametitle{Stripes Approach to Word Co-Occurrence}
  \footnotesize
  \begin{algorithm}[H]
    \Class{Mapper}{
      \Method{Map(docid \emph{a}, doc \emph{d})}{
        \ForEach{term w $\in$ doc d} {
          H$\ \leftarrow\ $ new Map$<$term$\ \rightarrow\ $ count$>$\\
          \ForEach{term u $\in$ \Near{(w)}}{
            H\{u\}$\ \leftarrow\ $ H\{u\}$\ +\ $1
          }
          \Emit{(term w, stripe H)}
        }
      }
    }
    \Class{Reducer}{
      \Method{Reduce(term w, stripes $[H_1,H_2,\ldots]$)}{
        $H_f\leftarrow\ $ new Map$<$term$\ \rightarrow\ $ count$>$\\
        \ForEach{stripe H $\in$ stripes $[H_1,H_2,\ldots]$} {
          \Sum{($H_f$, H)}
        }
        \Emit{(term w, stripe $H_f$)}
      }
    }
  \end{algorithm}
\end{frame}

\begin{frame}
  \frametitle{Potential Issues}
  \begin{itemize}
  \item<1,2> Memory and CPU utilization will increase
  \item<2> Emit less intermediate data at the cost of more expensive serialization
  \item<3> Don't forget your monitoring!
  \item<4> In-mapper combining and the flush technique may be used
    with the stripes approach
  \end{itemize}
\end{frame}

\section{\scshape Wrapping Up}

\subsection{Other Resources}
\begin{frame}{Other Resources}
  \begin{itemize}
  \item \emph{Data-Intensive Text Processing with MapReduce} by Jimmy
    Lin and Chris Dyer
  \end{itemize}
  (\url{http://lintool.github.com/MapReduceAlgorithms/index.html})
\end{frame}

\subsection{Contact Information}
\begin{frame}{Contact Information}
  \begin{itemize}
  \item E-mail: \href{mailto:andrew@andrewjamesjohnson.com}{andrew@andrewjamesjohnson.com}
  \item Slides are available at
    \href{http://andrewjamesjohnson.com/talks/}{my website \url{http://andrewjamesjohnson.com/talks/}} and
    \href{https://github.com/ajsquared/hadoop-slides}{on GitHub (\url{https://github.com/ajsquared/hadoop-slides})}
  \end{itemize}
\end{frame}

\subsection{Questions?}
\begin{frame}
  \Huge
  Questions?
\end{frame}


\end{document}
