%%%%%%%%%%%%%%%%%%%%%%%%%%%%%%%%%%%%%%%%%%%%%%%%%%%%%%%%%%%% 
%% This Beamer template was created by Cameron Bracken.
%% Anyone can freely use or modify it for any purpose
%% without attribution.
%% 
%% Last Modified: January 9, 2009
%% 

\documentclass[xcolor=x11names,compress]{beamer}

%% General document %%%%%%%%%%%%%%%%%%%%%%%%%%%%%%%%%%
\usepackage{graphicx}
\usepackage{tikz, amsmath, amsfonts, amsthm, amssymb}
\usepackage{hyperref}
\hypersetup{urlcolor=cyan}
\usetikzlibrary{decorations.fractals}
%%%%%%%%%%%%%%%%%%%%%%%%%%%%%%%%%%%%%%%%%%%%%%%%%%%%%% 


%% Beamer Layout %%%%%%%%%%%%%%%%%%%%%%%%%%%%%%%%%%
\useoutertheme[subsection=false,shadow]{miniframes}
\useinnertheme{default}
\usefonttheme{serif}
\usepackage{palatino}

\setbeamerfont{title like}{shape=\scshape}
\setbeamerfont{frametitle}{shape=\scshape}

\setbeamercolor*{lower separation line head}{bg=DeepSkyBlue4} 
\setbeamercolor*{normal text}{fg=black,bg=white} 
\setbeamercolor*{alerted text}{fg=red} 
\setbeamercolor*{example text}{fg=black} 
\setbeamercolor*{structure}{fg=black} 

\setbeamercolor*{palette tertiary}{fg=black,bg=black!10} 
\setbeamercolor*{palette quaternary}{fg=black,bg=black!10} 

\renewcommand{\(}{\begin{columns}}
  \renewcommand{\)}{\end{columns}}
\newcommand{\<}[1]{\begin{column}{#1}}
  \renewcommand{\>}{\end{column}}
\setbeamercovered{transparent}
%%%%%%%%%%%%%%%%%%%%%%%%%%%%%%%%%%%%%%%%%%%%%%%%%% 




\begin{document}


%%%%%%%%%%%%%%%%%%%%%%%%%%%%%%%%%%%%%%%%%%%%%%%%%%%%%% 
%%%%%%%%%%%%%%%%%%%%%%%%%%%%%%%%%%%%%%%%%%%%%%%%%%%%%% 
\section{\scshape Introduction}
\begin{frame}
  \title{Improving MapReduce Job Performance through Better Algorithm Design}
  \author{
    Andrew Johnson\\
    {\it Explorys}\\
  }
  \date{
    \begin{tikzpicture}[decoration=Koch curve type 1] 
      \draw[DeepSkyBlue4] decorate{ decorate{ decorate{ (0,0) -- (3,0) }}}; 
    \end{tikzpicture}  
    \\
    \vspace{1cm}
    \today
  }
  \titlepage
\end{frame}

\begin{frame}{Who am I?}
  \begin{itemize}
  \item Software engineer at Explorys
  \item CWRU grad in Computer Science, May 2012
  \end{itemize}
\end{frame}

\subsection{Overview}

\begin{frame}{Overview}
  \begin{itemize}
  \item<1,2,3> You've got your cluster set up and configured well.
  \item<2,3> You've written your MapReduce jobs.
  \item<3> But they are just not as fast as you'd like
  \item<4> How do you take your MapReduce jobs to the next level?
  \end{itemize}
\end{frame}

\section{\scshape Know Thyself}

\subsection{Know Thyself}
\begin{frame}{Know Thyself}
  \begin{itemize}
  \item<1> Understand your data!
  \item<2,3> Understand the current performance characteristics of your jobs!
  \item<3> Understand how much harder you can push your cluster!
  \end{itemize}
\end{frame}

\begin{frame}{Understanding Your MapReduce Jobs}
  \begin{itemize}
  \item<1,2> You need to have some kind of monitoring in place (e.g. Ganglia)
  \item<2,3> Your first task is to find out what is limiting the
    performance of your job.
  \item<3> Watch resource utilization during the time the job is
    running; is it CPU, IO, or memory bound?
  \end{itemize}
\end{frame}

\begin{frame}{Disclaimer}
  \begin{itemize}
  \item<1,2> The algorithmic techniques in this presentation reduce IO
    utilization at the cost of increased CPU and memory utilization in
    the map tasks.
  \item<2> While trying them out, continue to monitor resource
    utilization. \bf{Bad things can happen if you don't!}
  \item<3,4> I cannot guarantee that what I'm about to show you will
    actually improve the performance of your jobs.
  \item<4> Treat implementing these techniques as a science experiment.
  \item<5> That said, I've had good results with these techniques, so
    let's get to the fun stuff!
  \end{itemize}
\end{frame}

\section{\scshape Local Aggregation}

\subsection{Overview}
\begin{frame}{Overview}
  
\end{frame}

\subsection{Pairs}

\subsection{Stripes}

\section{\scshape Map-Only Jobs?!?}

\subsection{Map-Only Jobs?!?}

\section{\scshape Wrapping Up}

\subsection{Other Resources}
\begin{frame}{Other Resources}
  \begin{itemize}
  \item \emph{Data-Intensive Text Processing with MapReduce} by Jimmy
    Lin and Chris Dyer
  \end{itemize}
  (\url{http://lintool.github.com/MapReduceAlgorithms/index.html})
\end{frame}

\subsection{Contact Information}
\begin{frame}{Contact Information}
  \begin{itemize}
  \item E-mail: \href{mailto:andrew@andrewjamesjohnson.com}{andrew@andrewjamesjohnson.com}
  \item Twitter: \href{http://twitter.com/theajsquared}{@theajsquared}
  \item Slides are available at \href{}{my website} and
    \href{https://github.com/ajsquared/hadoop-slides}{on GitHub (\url{https://github.com/ajsquared/hadoop-slides})}
  \end{itemize}
\end{frame}

\subsection{Questions?}
\begin{frame}
  \Huge
  Questions?
\end{frame}


\end{document}
